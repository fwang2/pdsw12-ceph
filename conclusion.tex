\section{Observations and Conclusions}
\label{sec:conclusion}

Ceph is built on the assumption that the underlying hardware components
are unreliable, with little or no redundancy and failure detection capability.
This assumption is not valid for high-end HPC centers like ours. We have
disabled replication for pools, we haven't measured and quantified
processing overhead and we do not know yet if this would be significant.

Ceph performs \textbf{metadata + data} journaling, which is fine for host
systems that have locally attached disks. However, this presents a problem in
DDN SFA10K-like hardware, where the backend LUNs are exposed as block devices
through IB over SRP protocol. The journaling write requires twice the
bandwidth compared to Lustre-like meta data-only journaling mechanism. For
Ceph to be viable in large-scale capability HPC environments like ours,
journaling operations will need further design and more engineering efforts.

In our earlier tests, we experienced large performance swings during different
runs, low read performance when transfer size is small, and I/O errors tend to
happen when system is under stress (more clients and large transfer sizes).
However, with systematic performance engineering and development efforts, we
have seen a steady improvement through different releases. As of now, Ceph
system on our testbed is able to perform close to 80\% of raw hardware
capability at RADOS level and close to 70\% at file system level. This is
still no comparison to Lustre yet, but by no means a small feat for such a
\textit{young} technology. It is, in fact, a very respectable level of
performance. 

%The current design on journaling write does present a challenge in
%our IB-switched storage hardware. As BTRFS and other backend file system
%mature, we are seeing promising signs for Ceph to take advantage for a
%better journaling design.

