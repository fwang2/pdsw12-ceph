\section{Introduction}

Oak Ridge Leadership Computing Facility (OLCF)\footnote{This research was
supported by, and used the resources of, the Oak Ridge Leadership Computing
Facility, located in the National Center for Computational Sciences at ORNL,
which is managed by UT Battelle, LLC for the U.S. DOE (under the contract No.
DE-AC05-00OR22725).} at Oak Ridge National Laboratory (ORNL) has a long history
of deploying and running very-large-scale high-performance computing (HPC)
systems. 
%The facility hosted
%the Jaguar supercomputer up until recently, which was upgraded to the Titan
%supercomputer. 
In order to satisfy the I/O demand of such supercomputers, OLCF
also hosts large-scale file and storage systems. Lustre has been OLCF's choice
as the distributed parallel file system for scratch I/O. 
%The latest such
%large-scale file systems is Spider II\cite{spider2}.  Spider II, deployed in
%2013, is designed to operate at over 1 TB/s peak aggregate I/O throughput and
%has a 40 PB of raw disk capacity. As the scratch 
%file system for Titan, a leadership computing platform,
%Spider II emphasizes \textit{capability} over \textit{capacity}. 
However, we
recognize that there are a wide variety of scientific 
%data and 
%emerging
workloads with different performance and data access requirements (e.g., RESTful
interface, S3-like API, cloud solution integration) which might not be
efficiently serviced by Lustre. 
%OLCF is constantly evaluating new and emerging
%file and storage system technologies.  

%\textit{FEIYI: Are the 80\% and 70\% numbers still valid?}
This paper presents our evaluation and findings on block I/O and file system
performance of Ceph for scientific HPC environments. Through systematic
experiments and tuning efforts, we observed that Ceph can perform close to
70\% of raw hardware bandwidth at object level and about 62\% of at file
system level. We also identified that Ceph's metadata plus data journaling
design is currently a weak spot as far as our particular storage
infrastructure is concerned.

This rest of the paper is organized as follows. Section~\ref{sec:background}
provides an overview to Ceph and its architectural components;
Section~\ref{sec:testbed} gives an overview on our general test and evaluation
methodology as well as testbed environment; Following it,
Section~\ref{sec:baseline} establishes the baseline performance and
expectations for all critical components on the data path;
Section~\ref{sec:ceph-initial} discusses our early runs, results, and issues
bottom up: from middle tier RADOS object layer to the file system-level
performance. In Section~\ref{sec:ceph-tuning} we highlight the investigative
and tuning effort, we compare results before and after, and how the process
eventually bring the system performance to a respectable state. Finally,
Section~\ref{sec:conclusion} summarizes our findings, observations, and future
works.
