
% Also note that the "draftcls" or "draftclsnofoot", not "draft", option
% should be used if it is desired that the figures are to be displayed in
% draft mode.
%
\documentclass{sig-alternate}

\usepackage{microtype}
\usepackage{comment}
\usepackage{fancyvrb}
\usepackage{comment}


% set typewrite font to be small
\fvset{fontsize=\small}

% *** CITATION PACKAGES ***
%
\usepackage{cite}
\usepackage{graphicx}
\graphicspath{{./data/}{./figs/}}
\DeclareGraphicsExtensions{.pdf,.jpg,.png}


\usepackage{mdwmath}
\usepackage{mdwtab}

\usepackage{caption}
\DeclareCaptionType{copyrightbox}
\usepackage[font=footnotesize]{subfig}
\usepackage{url}
\usepackage[colorlinks = true,%
            linkcolor = blue,%
            urlcolor  = blue,%
            citecolor = blue,%
            anchorcolor = blue]{hyperref}

\makeatletter
\g@addto@macro\@verbatim\small
\makeatother

\usepackage{authblk}
\renewcommand\Affilfont{\normalsize}
\renewcommand\Authfont{\normalsize}
\author[1]{Feiyi Wang}
\author[2]{Mark Nelson}
\author[1]{Sarp Oral}
\author[1]{Scott Atchley}
\author[2]{Sage Weil}
\author[1]{Brad Settlemyer}
\author[1]{Blake Caldwell}
\author[1]{Jason Hill}

\affil[1]{Oak Ridge National Laboratory, Oak Ridge, Tennessee 37831}
\affil[2]{Inktank Inc., Los Angeles, CA 90017}

\begin{document}
\title{Performance and Scalability Evaluation of the \\Ceph Parallel File System}

\maketitle

\begin{abstract}

Ceph is designed to be a reliable and scalable fault tolerant parallel file
system with a number of intriguing features. In collaboration with its major
developer, Inktank Inc, we studied the feasibility of using Ceph for future
HPC storage deployment. This paper presents our experiments, results and
observations from mostly performance and scalability perspective. Our work
made two unique contributions. First, this evaluation is performed under a
realistic setup using state-of-the-art backend storage (Data Direct Network's
SFA10K),  although typical in large-scale HPC environment, but atypical for
average commercial off-the-shelf setup.  Second, our path of investigation,
tuning efforts, and findings made a direct contribution to Ceph's development,
it should also benefit community at large. Throughout the evaluation, we can
see that Ceph is under fast-paced development and maturing with great promise. 

\end{abstract}

\section{Introduction}

Oak Ridge Leadership Computing Facility (OLCF) at Oak Ridge National Laboratory
(ORNL) has a long history of deploying and running very-large-scale
high-performance computing (HPC) systems~\cite{top500-ornl}. The facility
hosted the Jaguar supercomputer up until recently, which was upgraded to the
Titan supercomputer. In order to satisfy the I/O demand of such supercomputers
(as well as, other OLCF computational, analysis, and visualization clusters),
OLCF also hosts large-scale file and storage systems. Lustre has been OLCF's
choice as the distributed parallel file system.  The latest incarnation of such
large-scale file systems hosted at OLCF is Spider II\cite{spider2}. Spider II
is the forklift-upgrade of its predecessor, Spider, which was deployed in 2008.
Spider II, deployed in 20013, is designed to operate at over 1 TB/s peak
aggregate I/O throughput and has a 40 PB of raw disk capacity. As the scratch
file system connected to a leadership computing platform at the scale of Titan,
Spider II emphasizes \textit{capability} over \textit{capacity}. However, we
also recognize that there are a wide variety of scientific data and workloads
with different performance and data access requirements (e.g., RESTful
interface, S3-like API, cloud solution integration) which might not be
efficiently serviced by Lustre. OLCF is constantly evaluating new and emerging
file and storage system technologies.  

Ceph\cite{Weil:2006:Ceph} originated from Sage Weil's PhD research at UC Santa
Cruz at 2007 and it was designed to be a reliable, scalable fault-tolerant
parallel file system.  Since then, Ceph has been open-sourced and Inktank is
now the major developer behind it.  In comparison to other parallel file
systems, Ceph has a number of distinctive features:

\begin{itemize}
 
\item Ceph has an intelligent and powerful data placement mechanism, known as
  CRUSH. The CRUSH algorithm allows a client to pre-calculate object
  placement and layout while taking into consideration of failure domains and
  hierarchical storage tiers.
  
  \item From the start, Ceph's design anticipated managing meta data and the
  name space with a cluster of meta data servers. It utilized a dynamic subtree
  partitioning strategy to continuously adapt meta data distribution to current
  demands.

  \item Ceph's design assumes that the system is composed of unreliable
  components; fault-detection and fault-tolerance (e.g., replication) are the
  norm rather than the exception. This is in line with the expectations and
  future directions of Exascale computing.

  \item Ceph is built on top of a unified object management layer, RADOS. Both
  meta data and the file data can take advantage of this uniformity. On top of
  RADOS, Ceph build and project a host of other features such as RESTful
  interface, S3 and Swift-compliant API, cloud integration.

  \item Most of the Ceph processes reside in user-space. Generally speaking,
this makes the system easier to debug and maintain. The client-side support has
long been integrated into Linux mainline kernel, which eases the deployment and
out-of-box experience.

\end{itemize}

As interesting and feature-rich as Ceph may be,  scalability and performance
are our top priorities due to the unique requirements our environment.  As part
of this study, we set up a dedicated testbed within OLCF to understand and
evaluate Ceph.  Our goal was to investigate the feasibility of using the Ceph
for our future HPC storage deployments. This paper presents our experience,
results, and observations.  While evaluating our results, please keep in mind
that Ceph is still a relatively \textit{young} parallel file system and its
code base is changing rapidly. In between releases, we often experienced
different stability and performance outcomes.  We will try to make clear in the
writing when such changes occurred.

This paper is organized as follows. Section~\ref{sec:testbed} gives an
overview on our general test and evaluation methodology as well as testbed
environment; Following it, Section~\ref{sec:baseline} establishes the baseline
performance and expectations for all critical components on the data path;
Section~\ref{sec:ceph-initial} discusses our early runs, results, and issues
bottom up: from middle tier RADOS object layer to the file system-level and
then meta data performance. In Section~\ref{sec:ceph-tuning} we highlight the
investigative and tuning effort, we compare results before and after, and how
the process eventually bring the system performance to a respectable state.
Finally, Section~\ref{sec:conclusion} summarizes our findings, observations,
and future works.

\section{Background}
\label{sec:background}

Ceph\cite{Weil:2006:Ceph} is a distributed storage system designed for
scalability, reliability, and performance.  The system is based on a
distributed object storage service called RADOS (reliable autonomic distributed
object store) that manages the distribution, replication, and migration of
objects.  On top of that reliable storage abstraction Ceph builds a range of
services, including a block storage abstraction (RBD, or rados block device)
and a cache-coherent distributed file system (CephFS).

Data objects are distributed across Object Storage Devices (OSD), which refers
to either physical or logical storage unit, using CRUSH\cite{Weil:2006:Crush},
a deterministic hashing function that allows administrators to define flexible
placement policies over a hierarchical cluster structure (e.g., disks, hosts,
racks, rows, datacenters).  The location of objects can be calculated based on
the object identifier and cluster layout (similar to consistent
hashing~\cite{karger1997consistent}), thus there is no need for a metadata
index or server for the RADOS object store.
%Further, because CRUSH provides an authoritative view of data placement,
%storage daemons can coordinate directly to handle data replications,
%recovery, or object migration in the face of failure or cluster topology
%changes.  
A small cluster of monitors (ceph-mon daemons) use Paxos to provide consensus
on the current cluster layout, but do not need to explicitly coordinate
migration or recovery activities.

CephFS builds a distributed cache-coherent file system on top of the object
storage service provided by RADOS.  Files are striped across replicated
storage objects, while a separate cluster of metadata servers (ceph-mds
daemons) manage the file system namespace and coordinate client access to
files.  

Ceph metadata servers store all metadata in RADOS objects, which provides a
shared, highly-available, and reliable storage backend.  Unlike many other
distributed file system architectures, Ceph also embeds inodes inside
directories in the common case, allowing entire directories to read from RADOS
into the metadata server cache or prefetched into the client cache using a
single request.

Client hosts that mount the file system communicate with metadata
servers to traverse the namespace and perform file I/O by reading and writing
directly to RADOS objects that contain the file data.  The metadata server
cluster periodically adjusts
the distribution of the namespace across the MDS cluster by migrating
responsibility for arbitrary subtrees of the hierarchy between a dynamic pool
of active ceph-mds daemons.  This dynamic subtree
partitioning~\cite{Weil:2004:dynamic} strategy is both adaptive and highly
scalable, allowing additional metadata server daemons to be added or removed
at any time, making it ideally suited both for large-scale workloads with
bursty workloads or general purpose clusters whose workloads grow or contract
over time.

\begin{comment}
In comparison to other parallel file
systems, Ceph has a number of distinctive features:

\begin{itemize}
 
\item Ceph has an intelligent and powerful data placement mechanism, known as
  CRUSH. The CRUSH algorithm allows a client to pre-calculate object
  placement and layout while taking into consideration of failure domains and
  hierarchical storage tiers.
  
  \item From the start, Ceph's design anticipated managing meta data and the
  name space with a cluster of meta data servers. It utilized a dynamic subtree
  partitioning strategy to continuously adapt meta data distribution to current
  demands.

  \item Ceph's design assumes that the system is composed of unreliable
  components; fault-detection and fault-tolerance (e.g., replication) are the
  norm rather than the exception. This is in line with the expectations and
  future directions of Exascale computing.

  \item Ceph is built on top of a unified object management layer, RADOS. Both
  meta data and the file data can take advantage of this uniformity. On top of
  RADOS, Ceph build and project a host of other features such as RESTful
  interface, S3 and Swift-compliant API, cloud integration.

  \item Most of the Ceph processes reside in user-space. Generally speaking,
this makes the system easier to debug and maintain. The client-side support has
long been integrated into Linux mainline kernel, which eases the deployment and
out-of-box experience.

\end{itemize}
\end{comment}

While we are interested in Ceph for its ability to support alternative
workloads that are not easily accomplished with Lustre, in this study we are
investigating the use Ceph for future large-scale scientific HPC storage
deployments. This paper presents our experience, results, and observations in
measuring the scalability and performance of Ceph.  

%Additionally, Ceph is under
%rapid development, and in between releases, we often
%experienced different stability and performance outcomes.  
%We will try to make
%clear in the writing when such changes occurred.
%As interesting and feature-rich as Ceph may be,  scalability and performance
%are our top priorities due to the unique requirements our environment.  As part
%of this study, we set up a dedicated testbed within OLCF to  evaluate Ceph.
%Our goal was to investigate the feasibility of using the Ceph for future
%large-scale scientific HPC storage deployments. This paper presents our
%experience, results, and observations.  While evaluating our results, please
%keep in mind that Ceph is still a relatively \textit{young} parallel file
%system and its code base is changing rapidly. In between releases, we often
%experienced different stability and performance outcomes.  We will try to make
%clear in the writing when such changes occurred.

%This paper is organized as follows. Section~\ref{sec:testbed} gives an
%overview on our general test and evaluation methodology as well as testbed
%environment; Following it, Section~\ref{sec:baseline} establishes the baseline
%performance and expectations for all critical components on the data path;
%Section~\ref{sec:ceph-initial} discusses our early runs, results, and issues
%bottom up: from middle tier RADOS object layer to the file system-level and
%then meta data performance. In Section~\ref{sec:ceph-tuning} we highlight the
%investigative and tuning effort, we compare results before and after, and how
%the process eventually bring the system performance to a respectable state.
%Finally, Section~\ref{sec:conclusion} summarizes our findings, observations,
%and future works.

\section{Testbed Environment Description}

We used Data Direct Networks' (DDN) SFA10K as the storage backend during this
evaluation. It consists of 200 SAS drives and 280 SATA drives, organized into
various RAID levels by two active-active RAID controllers. The exported RAID
groups by these controllers are driven by four hosts.
Each host has two InfiniBand (IB) QDR connections to the storage backend.
We used a single dualport Mellanox connectX IB card per host.
By our calculation, this setup can saturate SFA10K's maximum theoretical
throughput (\textasciitilde 12 GB/s). The connection diagram is illustrated in
Figure~\ref{fig:ddn-sfa10k}.

\begin{figure}[htb]
\centering
\includegraphics[width=3in]{figs/sfa10k}
\caption{DDN SFA10K hardware and host connection diagram}
\label{fig:ddn-sfa10k}
\end{figure}


Our Ceph testbed employs a collection of testing nodes. These nodes and their
roles are summarized in Table~\ref{tbl:ceph-test-nodes}. In the following
discussion, we use ``servers'', ``osd servers'', ``server hosts''
interchangably. We will emphasize with ``client'' prefix when we want to
distinguish it from above.

\begin{table}[!t]
\centering
\caption{Support nodes involved in Ceph testbed}
\label{tbl:ceph-test-nodes}
    \begin{tabular}{ll}
    \hline
    Node & Role \\
    \hline
    tick-mds1 & Ceph monitor node \\
    spoon46 & Ceph MDS node \\
    tick-oss[1-4] & Ceph OSD servers \\
    spoon28-31, spoon37-41 & Ceph client nodes \\
    \hline
    \end{tabular}
\end{table}

All hosts  (client and servers) were configured with Redhat 6.3 and kernel
version 3.5.1 initially, and later upgraded to 3.9 (rhl-ceph image), Glibc
2.12 with syncfs support, locally patched.  We used the Ceph 0.48 and 0.55
release in the initial tests, upgraded to 0.64 and then to 0.67RC for a final
round of tests.

%For a complete a list of hosts that are running ceph images, one can execute:

%\begin{Verbatim}
%$ grep "rhel6-ceph" /etc/gedi/MAC.info
%\end{Verbatim}


\section{Establishing A \\Performance Baseline}
\label{sec:baseline}

\subsection{Block I/O over Native IB} 
\label{sec:block-io}

The lowest layer in the system are the exposed LUNs to the server hosts from
the RAID controllers.  These LUNs are configured as a RAID 6 8+2 array (8 data
disks and 2 \textit{P} and \textit{Q} disks). In prior tests we observed
that a 7.2K RPM SAS disk can perform at \textasciitilde140 MB/s for 128 kB
sequential I/O requests and a 7.2K RPM SATA disk can do \textasciitilde36 MB/s.
For these tests, all disk caches were turned off. Therefore, in 8 data disk
RAID group with 1 MB sequential I/O requests (each disks sees 128 kB chunks of
the request) the expected aggregate performance would be \textasciitilde1.12
GB/s for a SAS RAID group and \textasciitilde288 MB/s for a SATA RAID group, if
there are no caches along the I/O path.  With the caches turned on, especially
the write-back cache on the DDN RAID controller, we observe a significant
performance improvement. In our day-to-day HPC operations we run our storage
systems with write-back cache turned on on RAID controllers, if and only if,
there is a method of cache-mirroring between the two active-active RAID
controllers. With the write-back cache on we observe \textasciitilde1.4 GB/s
per RAID 6 8+2 SAS group and  \textasciitilde955 MB/s per RAID 6 8+2 SATA
group, for 1 MB sequential I/O.  These two performance numbers will be used as
a baseline in our study and they are presented in
Table~\ref{tbl:block-io-baseline}.  For these tests, the 8+2 RAID 6 groups are
exported to the server hosts using SCSI RDMA Protocol (SRP) over IB protocol
(which is the default for our normal HPC configurations). For a SATA disk group
this translates into roughly 120 MB/s/disk and for a SAS disk group it is 175
MB/s/disk. As can be seen, SATA RAID groups benefit from caching more than SAS
groups. 

%For above tests the RAID groups are exported over physical 4x IB QDR links.
%These links can provide up to 40 Gbits/s signalling rate and 32 Gbits/s actual
%data rate (reduction is due to 8/10 encoding scheme). Therefore, each IB QDR
%link can perform at 3.2 GB/s, while in practice we see around 3 GB/s as the
%best case. 

%As mentioned earlier, we had 280 SATA disks and 200 SAS disks. Organizing these
%into RAID 6 8+2 groups yields 28 SATA and 20 SAS RAID 6 8+2 groups. We had
%4 server hosts in our testbed and evenly distributing these on to 4 server
%hosts translates into 7 SATA LUNs and 4 SAS LUNs per server host. 

%Next, we established the baseline performance for a single server host. We
%exercised all seven SATA LUNs concurrently form a single server host at the
%block-level. This resulted in an aggregate performance of roughly 2.6 GB/s.
%Repeating the same test using only four SAS LUNs yielded around 2.8 GB/s
%aggregate performance for 1 MB sequential I/O. These two numbers will be used
%as single server host aggregate baseline performance for this study. These are
%also presented in Table~\ref{tbl:block-io-baseline}. 

It is worth mentioning that we used our in-house developed synthetic benchmark,
\textit{fair-lio} for all our block-level testing.  Fair-lio has better
sequential I/O characteristics compared to commonly used XDD benchmark and it
is asynchronous and based on the Linux \verb!libaio! library.


\begin{table}[htb]
\centering
\caption{Baseline block I/O performance summary}
\label{tbl:block-io-baseline}

\begin{tabular}{ l | l }
    \hline
    SAS single LUN sequential read & ~1.4 GB/s \\
    SATA single LUN sequential read & ~955 MB/s \\
    Single host with four SAS LUNs & ~ 2.8 GB/s \\
    Single host with seven SATA LUNs & ~ 2.6 GB/s \\
    \hline
\end{tabular}
\end{table}

When exercising all 28 SATA LUNs from all four server hosts in parallel, we
observed an 11 GB/s aggregate performance. The same level of performance was
observed using all 20 SAS LUNs from four server hosts in parallel. Comparing
these to the advertised peak performance of 12 GB/s of the DDN SFA10K, we
concluded that our test setup is well configured to drive the backend disk
system at close to full performance and we are limited by the RAID controller
performance. Going forward in this study, 11 GB/s will be used the peak
baseline for the entire test configuration.


\subsection{Establishing an IP over IB Baseline Performance}

Ceph uses the BSD Sockets interface in \texttt{SOCK\_STREAM} mode (i.e., TCP).
Because our entire testbed is built on IB QDR fabric (including client to
server host connections over a 108-port Mellanox IB QDR switch), we used IP
over IB (IPoIB) for networking\footnote{As of writing of this paper, Inktank
is investigating using rsockets to improve performance with IB fabric}. Through
simple \verb!netperf! tests, we observed that a pair of hosts connected by IB
QDR using IPoIB can transfer data at 2.7 GB/s (the hosts are tuned per
recommendations from Mellanox). With all four server hosts (OSD servers), we
expect the aggregate throughput to be in the neighborhood of 11 GB/s.

%Unfortunately there was not enough time to do more detailed analysis of the
%network performance. However, as we observed later, RADOS is performing no
%more than 8 GB/s driven by four server hosts. This confirms that we have
%provisioned enough network bandwidth. In other words, IP over IB is not a
%bottleneck in this case.


\section{Ceph Evaluation: Initial Results}
\label{sec:ceph-initial}

\subsection{Ceph RADOS Scaling}


RADOS is a reliable distributed object store, the foundational component for
CephFS file system. There are two types of scaling tests we are interested at
the RADOS layer:

\begin{itemize}
  \item scaling on the number of OSD servers
  \item scaling on the number of OSDs per OSD server
\end{itemize}

Our system setup poses some limitations on the scalability tests we wanted to
perform. In particular, we currently had four OSD servers, eight clients, and
eleven OSD servers per client. The scaling tests, therefore, will be within
these constraints.

We used the open-source RADOS Bench tool, developed by Inktank, to perform our
initial performance analysis of the underlying RADOS layer.  RADOS Bench simply
writes out objects to the underlying object storage as fast as possible, and
then later reads those objects in the same order as they were written.

We observed that using two or more client processes and many concurrent
operations are important when performing these tests.  We tested eight client processes 
with 32 concurrent 4 MB objects in flight each. We created a pool for
each RADOS Bench process to ensure that object reads come from independent pools
(RADOS Bench is not smart enough to ensure that objects are not read by multiple
processes and thus possibly cached).  A sync and flush is performed on every
node before every test to ensure no data or metadata is in cache.  All tests
were run with replication set to one.  The backend file systems were XFS,
BTRFS and EXT4 file systems were not tested at this time.


\subsubsection{Scaling on number of OSDs per server}

In the following test, a single Ceph host drives $n$ OSDs, where $n$ increases
from one to eleven. The result is illustrated in Figure~\ref{fig:osd-scale}.
We ran the test against a single client with four concurrent processes. In this
test case, we observe that the OSD server exhibits near linear scalability up to
nine OSDs, and is still trending upwards at eleven OSDs. This suggests that we
have not reached the saturation point yet. Additional testing would require
provisioning more OSDs on the SFA10K backend.


\begin{figure}[!t]
\centering
\includegraphics[width=3in]{data/rados_osd}
\caption{RADOS scaling on number of OSDs}
\label{fig:osd-scale}
\end{figure}

\begin{figure}[!t]
\centering
\includegraphics[width=3in]{data/rados_server}
\caption{RADOS scaling on number of servers}
\label{fig:oss-scale}
\end{figure}

\subsubsection{Scaling on number of OSD servers}

In this test, we exercise OSD servers from one to four, driven by four hosts
each with four RADOS Bench process. Each additional OSD server
adds eleven more OSDs into play. We observe that Ceph exhibits linear scaling with
regard to number of servers as well, at least in the given set of servers.
However, the peak performance we are seeing is about the half of what are
expecting from the SFA10K (compare to the baseline block I/O perforamnce number
presented in Section~\ref{sec:block-io}).

For writes, the lost performance is attributed to the way Ceph performs
journaling:
Ceph does not support meta-data only journaling, therefore every write is the
equivalent of a double-write: once to the data device, once to the journaling
device. This effectively cuts the observed system bandwidth in half. That said,
it does not explain the read performance -- it is a little better than write,
but still far from the theoretical maximum.

\subsection{Ceph File System Performance}

We used the synthetic IOR benchmark suite for file system level performance
and scalability test.  The particular parameter setup is show in Table
\ref{tbl:ior}. Each client node has 6 GB of physical memory, the block size is
set so as to mitigate cache effects. In addition, the test harness program
issues the following commands at the beginning of each test:


\begin{table}[tb]
\caption{IOR parameter setup}
\label{tbl:ior}
\centering
\begin{tabular}{p{0.8in} | p{2in}}
    \hline
    IOR parameter & Note \\ \hline
    \verb!-F! & file per process \\ \hline
    \verb!-a POSIX! & use POSIX API \\ \hline
    \verb!-w -r -C! & do both write and read test, \verb!-C! is to change task
        ordering for read back so it will not read from the write cache. \\
        \hline
    \verb!-i 3 -d 5! & 3 iterations and delay 5 seconds betewen iterations \\ \hline  
    \verb!-e! & perform \verb!fsync()! upon POSIX write close \\ \hline
    \verb!-b 8g or 16g! & the block size \\ \hline
    \verb!-t 4k to 4m! & the transfer size \\ \hline
    \verb!-o file! & mandatory test file  \\    
    \hline
\end{tabular}
\end{table}


\begin{Verbatim}[fontsize=\small]
# sync
# echo 3 | tee /proc/sys/vm/drop_caches
\end{Verbatim}


Here, 0 is the default value of \verb!drop_caches!; 1 is to free pagecaches, 2
is to free dentries and inodes, 3 is to free pagecache, dentries, and inodes.


Our first round of tests was less than ideal as we encountered various issues. For
the sake of completeness, we first summarize the results, then discuss
further tuning efforts and improvements.

The full permutation of IOR parameters were not explored due to I/O errors we encountered.
We were, however, able to record results in two extreme cases as far as
transfer size is concerned: 4 KB and 4 MB, using a fixed number of OSD servers
(4) and fixed block size (8 GB), the results are illustrated in Figure
\ref{fig:ior4k} and \ref{fig:ior4m}, we make the following observations:


\begin{figure*}[!t]

\centerline{\subfloat[4 KB transfer size]
{\includegraphics[width=3in]{data/ior_4k}
\label{fig:ior4k}}
\hfil
\subfloat[4 MB transfer size]
{\includegraphics[width=3in]{data/ior_4m}
\label{fig:ior4m}}
}% end of centerline
\caption{CephFS scalability test with IOR}

\end{figure*}


\begin{itemize}

  \item The small read (4 KB transfer size) performance is almost an anomaly
  -- we will investigate why it is so low compare to write performance and
  present improved results in Section~\ref{sec:improve-ior}.

  \item The large read (4 MB transfer size) performance is almost half of the
  RADOS read performance.
   
  \item The write performance is also about half of what we can obtain from
  RADOS Bench. When number of clients reaches 8, there is a significant
  performance drop as well. 

\end{itemize}


We will describe the efforts and results on performance improvement in the
following sections.


\section{Improving RADOS Performance}

\begin{figure}[h]
\centering
\includegraphics[width=3.5in]{parametric}
\caption{Evaluating parameter impact through sweeping test}
\label{fig:parametric}
\end{figure}


After the initial test results, we tried various combinations of tweaks
including changing the number of filestore op threads, putting all of the
journals on the same disks as the data, doubling the OSD count, and upgrading
Ceph to a development version which reduces the seek overhead caused by
\texttt{pginfo} and \texttt{pglog} updates on XFS (these enhancements are now
included as of the Ceph Cuttlefish release, v0.61).  The two biggest
improvements resulted from disabling CRC32c checksums and increasing the OSD
count on the server.  With these changes, we are seeing better results.

We ran a script written by Inktank for internal Ceph testing to perform
sweeps over Ceph configuration parameters to examine how different
tuning options affect performance on the DDN platform. The result of this
parameter probing is illustrated in Figure~\ref{fig:parametric}. Please refer
to Appendix E for explanations of these probed parameters.


As a result of this testing, we improved performance slightly by
increasing the size of various Ceph buffers and queues, enabling AIO journals,
and increasing the number of OSD op threads.


\subsection{Disable Cache Mirroring on Controllers}

During a second round of test performed by Inktank, we noticed a dramatic drop
on RADOS performance: even though write throughput on individual server met the
expectation, it did not scale across servers.

We spent a significant amount of time
investigating this phenomenon. Ultimately, we were able to replicate this finding
when running concurrent disk throughput tests directly on the servers without
Ceph involved. The second RAID processor on each DDN controller would max out when
three or more LUNs were written concurrently. It turns out the root of the problem
was a regression on DDN firmware update -- in particular, the cache
mirroring was not behaving as it should.\footnote{DDN recently released a new
firmware version and we were told the issue has been fixed. Unfortunately, we didn't get
a chance to verify it during our test cycle.}

%% SCOTT - is running with cache mirroring off and option for a production system or not?
%% What are the consequences? Did DDN eventually provide a fix? If yes, were we able to test
%% with it or not?

%% FEIYI: add footnote to clarify the issue.

\begin{figure}[htb]
\centering
\includegraphics[width=3.5in]{rados-after-ddn}
\caption{Evaluating RADOS bench after disabling cache mirroring}
\label{fig:rados-ddn-mirror-disabled}
\end{figure}


With cache mirroring disabled, write performance when using all four servers
improved dramatically, as illustrated in
Figure~\ref{fig:rados-ddn-mirror-disabled}. With BTRFS, for example, we hit over
5.5 GB/s from the clients.  When accounting for journal writes, that is over
11 GB/s to the disks and very close to what the DDN chassis is capable of doing. 
Unfortunately, read performance did not scale as well.


\subsection{Disable TCP autotuning}

During these tests, a trend that previously had been seen became more
apparent.  During reads, there were periods of high performance followed by
periods of low performance or outright stalls that could last for up to 20
seconds at a time.  After several hours of diagnostics, Inktank observed that
outstanding operations on the clients were not being shown as outstanding on
the OSDs.  This appeared to be very similar to a problem Jim Schutt at Sandia
National labs encountered with TCP autotuning in the Linux
kernel.\footnote{\url{http://marc.info/?l=ceph-devel&m=133009796706284&w=2}}
TCP auto tuning enables TCP window scaling by default and automatically
adjusts the TCP receive window for each connection based on link conditions
such as bandwidth delay product. We have observed this will make a notable
improvement on Ceph read performance, as the results shown in
Figure~\ref{fig:rados-tcp-auto-disabled}.


Luckily, the fix was fairly straight forward by issuing the following command on all nodes:

\begin{Verbatim}
     echo 0 | sudo tee /proc/sys/net/ipv4/tcp_moderate_rcvbuf
\end{Verbatim}

Recent versions of Ceph work around this issue by manually controlling the TCP
buffer size.  The testing at ORNL directly influenced and motivated the creation
of this feature!

\begin{figure}[htb]
\centering
\includegraphics[width=3.5in]{rados-after-ddn-tcptune}
\caption{Evaluating RADOS bench after TCP auto tuning disabled}
\label{fig:rados-tcp-auto-disabled}
\end{figure}



%\begin{figure}[htb]
%\includegraphics[width=5in]{rados-064-oss}
%\end{figure}

\subsection{Repeating RADOS Scaling Test}

We now repeated the previous RADOS scaling tests with these improvements in place.
The first test was done on a single node with RADOS Bench to see how close the
underlying object store could get to the node hardware limitations as the number
of OSDs/LUNs used on the node increased. Note all the tests performed were against
XFS-formatted storage.

%% SCOTT - the title in the figure needs to change IO to I/O

%% SCOTT - how do writes inc. journals exceed the client network max? Dual ports
%% on the single server?

\begin{figure}[htb]
\centering
\includegraphics[width=3.5in]{rados-064-osd}
\caption{RADOS Bench Scaling on \# of OSD, Ceph 0.64, 4 MB I/O, 8 Client Nodes}
\label{fig:rados-064-osd}
\end{figure}

In the single server case as shown in Figure~\ref{fig:rados-064-osd}, ``Writes
(including Journals)'' refers to how much data is actually being written out the
DDN chasis, and blue line is how much data the clients are writing.
We observe that performance gets very close to the hardware limits at roughly 9
OSDs per server and then mostly levels out.

We also repeated tests looking at RADOS Bench performance as the number of OSD
server nodes increases from one to four. The results are summarized in
Figure~\ref{fig:rados-064-oss}. As the number of nodes increases, performance
scales nearly linearly for both reads and writes.

%% SCOTT - why does the client network max scale in figure 12 but not figure 11?
%% Do they no measure the same thing (i.e. a single server)? If not, the text
%% needs to make it more clear.

\begin{figure}[htb]
\centering
\includegraphics[width=3.5in]{rados-064-oss}
\caption{RADOS Bench Scaling on number of servers, Ceph 0.64, 4 MB I/O, 8 client
nodes}
\label{fig:rados-064-oss}
\end{figure}



\section{Improving Ceph File System Performance}
\label{sec:improve-ior}

The initial stability issues mentioned in Section~\ref{sec:ior-initial} are
fixed by migrating from Ceph version 0.48/0.55 to 0.64, the latest stable version at the
time of writing this report.  Upgrading to the latest stable Ceph release
allowed us to run a full IOR parameter sweep for the first time since we
started evaluating the performance and scalability of the Ceph file system.
This is another sign of how much Ceph development is currently in flux.

Another fix introduced by Ceph version 0.64 was in pool creation.  The default
data pool used by previous Ceph version were set to 2x replication by mistake.
This potentially halved the write performance. With version 0.64 we explicitly
set the replication level to 1, which is the preferred value for a HPC
environment like ours running on high-end and reliable storage backend hardware
(e.g. DDN SFA10K).

Even with these two changes in place, less-than-ideal write performance and
very poor read performances were observed during our tests.  We also observed
that by increasing the number of IOR processes per client node, the read
performance degraded even further indicating some kind of contention either on
the clients or on the OSD servers.


\subsection{Disabling Client CRC32}

At this point, we were able to both make more client nodes available for Ceph
file system-level testing and also install a profiling tool called \verb!perf!
that is extremely useful for profiling both kernel and user space codes.
Profiling with \verb!perf! showed high CPU utilization on test clients due to
crc32c processing in the Ceph kernel client.  crc32 checksums can be disabled
by changing the CephFS mount options:

\begin{Verbatim}
mount -t ceph 10.37.248.43:6789:/ /mnt/ceph -o name=admin,nocrc
\end{Verbatim}


With client CRC32 disabled, we repeated the IOR tests. New results are shown in
in Figure~\ref{fig:ior-no-client-crc32}. 

\begin{figure}[htb]
\centering
\includegraphics[width=3.5in]{ior-client-no-crc32}
\caption{IOR test with disabling client-side CRC32}
\label{fig:ior-no-client-crc32}
\end{figure}

We observed that IOR write throughput increased dramatically and is now very
close and comparable to the RADOS Bench performance. Read performance continued
to be poor and continued to scale inversely with the increasing number of
client processes.  Please note that, since these tests were performed, Inktank
has implemented SSE4-based CRC32 code for Intel CPUs.  While any kernel based
CRC32 processing should have already been using SSE4 instructions on Intel
CPUs, this update will allow any user-land Ceph processes to process CRC32
checksums with significantly less overhead.

\subsection{Improving IOR Read Performance}

Deeper analysis with perf showed that there was heavy lock contention during
parallel compaction in the Linux kernel memory manager.  This behavior was first
observed roughly in the kernel 3.5 time frame which was the kernel
installed on our test systems.\footnote{For more information,
please refer to \url{http://lwn.net/Articles/517082/} and
\url{https://patchwork.kernel.org/patch/1338691/}.}

We upgraded our test systems with kernel version 3.9 and performed RADOS Bench
test.  The results were extremely positive and presented in
Figure~\ref{fig:rados-kernel}.


\begin{figure}[htb]
\centering
\includegraphics[width=3.5in]{rados-kernel-35vs39}
\caption{RADOS bench: Linux kernel version 3.5 vs. 3.9}
\label{fig:rados-kernel}
\end{figure}



As can be seen, with the 3.9 kernel, while there was a slight improvement on
write performance, read performance improved dramatically.  In addition to the
kernel change, Sage Weil from InkTank suggested increasing the amount of CephFS
client kernel read-ahead cache size as:

\begin{Verbatim}[samepage=true]
mount -t ceph 10.37.248.43:6789:/ /mnt/ceph -o
   name=admin,nocrc,readdir_max_bytes=4104304,readdir_max_entries=8192
\end{Verbatim}

%% SCOTT - is figure 15 IOR over Ceph and 14 is RADOS bench? If so, can we
%% mention that 15 shows the filesystem performance?

IOR results reflecting the read-ahead cache size change are presented in
Figure~\ref{fig:ior-kernel-39}.

\begin{figure}[htb]
\centering
\includegraphics[width=3.5in]{ior-kernel-39}
\caption{CephFS performance with kernel changes to 3.9, IOR with 4 MB transfer
size}
\label{fig:ior-kernel-39}
\end{figure}


By installing a newer kernel, increasing read-ahead cache size, and increasing the number of
client IOR processes, we were able to achieve very satisfactory I/O performance.


\subsection{Repeating the IOR Scaling Test}

As before, we ran IOR scaling tests with two cases: transfer size 4 KB and 4 MB.
These results are illustrated in Figure~\ref{fig:ior-064}. As expected, we saw
saw  improved read and write performance. These new read and write performance are in
line with observed RADOS bench performance.

%% SCOTT - Can we use the same Y axis for these figures? Any comment from Inktank
%% as to why 4 KB performance is better than 4 MB performance?

\begin{figure}[htb]
\centering
\includegraphics[width=3.5in]{ior-064-4k}
\includegraphics[width=3.5in]{ior-064-4m}
\caption{IOR Scaling Test: 4 KB and 4 MB transfer size}
\label{fig:ior-064}
\end{figure}

Throughout our IOR testing, we observed that the average write throughput is 
lower than the maximum.  This behavior was observed during other tests as well,
indicating that we may have periods of time where write throughput is
temporarily degrading.  Despite these issues, performance generally seems to be
improving with respect to increasing number of clients.  Writes seem to be topping out at
around 5.2 GB/s (which is roughly what we would expect).  Reads seem to be
topping out anywhere from 5.6-7 GB/s, however it is unclear if read performance
would continue scaling with more clients and get closer to ~8 GB/s  we
obervered with RADOS Bench.


\section{Observations and Conclusions}
\label{sec:conclusion}

Ceph is built on the assumption that the underlying hardware components are
unreliable, with little or no redundancy and failure detection capability.
This assumption is not valid for high-end HPC centers like OLCF. We have
disabled replication for pools, haven't measured and quantified processing
overhead, and we do not know yet if this would be significant. Investigating
this remains as a future work. 

Ceph performs \textbf{metadata + data} journaling, which is fine for host
systems that have locally attached disks. However, this presents a problem in
DDN SFA10K-like hardware, where the backend LUNs are exposed as block devices
through IB over SRP protocol. The journaling write requires twice the
bandwidth compared to Lustre-like meta data-only journaling mechanism. For
Ceph to be viable in large-scale capability HPC environments like OLCF,
journaling operations need further design and more engineering efforts.

In our earlier tests, we experienced large performance swings during different
runs, low read performance when transfer size is small, and I/O errors tend to
happen when system is under stress (more clients and large transfer sizes).
However, with systematic performance engineering and development efforts, we
have seen a steady improvement through different releases. As of now, Ceph
system on our testbed is able to perform close to 80\% of raw hardware
capability at RADOS level and close to 70\% at file system level (after
accounting for journaling overheads) . This is still no comparison to Lustre
yet, but by no means a small feat for such a \textit{young} technology. It is,
in fact, a very respectable level of performance. 

%The current design on journaling write does present a challenge in
%our IB-switched storage hardware. As BTRFS and other backend file system
%mature, we are seeing promising signs for Ceph to take advantage for a
%better journaling design.


\section*{Acknowledgments}

The authors would like to thank Galen Shipman of ORNL for initiating the
evaluation effort, and Greg Farnum of Inktank for providing early support.


%The authors would like to thank Galen Shipman of ORNL for initiating the
%evaluation effort, and Greg Farnum of Inktank for providing early support.

\pagebreak
\bibliography{paper}
\bibliographystyle{abbrv}
\end{document}


